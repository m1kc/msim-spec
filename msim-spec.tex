\documentclass[a4paper,12pt]{book}
%\documentclass[draft,a4paper,12pt]{book}
\usepackage[utf8]{inputenc}
\usepackage[russian]{babel}
\usepackage{amsmath}
\usepackage{amsfonts}
\usepackage{amssymb}

\title{\textbf{Спецификации протокола mSIM}}
\author{Мусатов М.Д. \and Солкин И.В.}

\begin{document}

\maketitle

\chapter{В целом}

\section{О протоколе mSIM}

Что такое mSIM? Если в двух словах, то это лёгкий расширяемый протокол для обмена текстовыми сообщениями. mSIM был создан с прицелом на нетребовательность к ресурсам, масштабируемость и простоту реализации.

\section{Суть протокола}

Итак. В прокотоле mSIM имеются два главных действующих лица --- клиент и сервер. Что характерно, клиент подключается к серверу через сокет с использованием протокола TCP (рекомендуемый порт --- 3215). После этого начинается процесс обмена информацией, которая передаётся в виде пакетов.

Пакет mSIM по своей структуре очень прост. Он состоит из следующих друг за другом трёх строк в кодировке UTF-8 --- эти строки называются полями. В начале каждой строки идут два байта, обозначающие её длину (в прямом порядке, например, 8 записывается как \texttt{0x00 0x08}), за ними --- собственно значащие байты\footnote{Больше всего повезло программистам на Java, потому что они могут прозрачно работать с такими строками через методы DataInputStream.readUTF() и DataOutputStream.writeUTF() для чтения и записи соответственно.}. Названия и назначения полей приведены ниже.

\begin{enumerate}
\item \texttt{Destination}: Логин получателя (для исходящих пакетов) или отправителя (для входящих). Если пакет адресован самому серверу, здесь должно стоять имя \texttt{SERVER}.
\item \texttt{Type}: Тип пакета.
\item \texttt{Content}: Содержимое пакета.
\end{enumerate}

Поле \texttt{Content} содержит собственные поля, количество и назначение которых зависит от типа пакета. Эти поля отделяются друг от друга символом вертикальной черты: <<\texttt{|}>>. Некоторые типы пакетов предусматривают передачу информации в формате INI.

Пакеты не имеют заголовков, маркеров и никак не отделяются друг от друга. Типы пакетов и информацию, которую они в себе несут, мы подробно рассмотрим далее.

Что-то ещё? Да, сервер разрывает соединение в случае, если клиент в течение 120 секунд не подаёт признаков жизни. Поэтому не забывайте периодически отправлять пинги (см. п. \ref{ping}).

\section{Технологии}

Для реализации клиента mSIM точно потребуется поддержка следующих технологий:

\begin{itemize}
	
\item Сокеты
\item Строки в UTF-8
\item Разбиение строки на поля по заданному разделителю
\item Парсер формата INI
	
\end{itemize}

Если говорить про реализацию на Java SE, то первые три возможности изначально встроены в язык, а для четвёртой можно использовать библиотеку BinGear, которая используется и в официальном клиенте mSIM.

\chapter{В деталях}

\section{Соглашение об обозначениях}

Обозначения обозначаются обозначениями в соответствии с тем, что они обозначают. Статью потом допилю.

\section{Базовые типы пакетов}

\subsection{<<Успех>>}

Пакет с типом \texttt{success} оповещает об успешном завершении операции. Как правило, в поле \texttt{Content} вписано её название.

\subsection{<<Неудача/отказ>>}

Пакет с типом \texttt{fail} оповещает о том, что запрошенную операцию по какой-то причине не удалось выполнить. Как правило, в поле \texttt{Content} вписано её название.

\subsection{<<Нет необходимости>>}

Пакет с типом \texttt{wtf} оповещает о том, что операция не была совершена, поскольку в ней нет необходимости. Как правило, в поле \texttt{Content} вписано её название.

\subsection{<<Внутренняя ошибка>>}

Пакет с типом \texttt{error-internal} оповещает о том, что операция не была закончена из-за неожиданного исключения. Поле \texttt{Content}, как правило, пустое.

\subsection{<<Неизвестный тип пакета>>}

Пакет с типом \texttt{error-unknown-type} оповещает о том, что другой стороне незнаком этот тип пакета и он (пакет), вероятно, ошибочный. В поле \texttt{Content} вписан тип полученного пакета.

\subsection{<<Не поддерживается/ещё не \mbox{реализовано}>>}

Пакет с типом \texttt{error-not-implemented} оповещает о том, что операция не была выполнена из-за того, что другая сторона её не поддерживает. Как правило, в поле \texttt{Content} вписано её название.

\section{Основные возможности протокола}

\subsection{Проверка связи}
\label{ping}

Используется для (сюрприз!) проверки связи и просто для поддержания соединения в активном состоянии.

Типичный случай использования:

{\ttfamily
<{}< SERVER: ping: <<>>

>{}> SERVER: ping-response: <<>>
}

\subsection{Авторизация}
\label{auth}

Используется для авторизации на сервере.

Процедура в общем случае выглядит так:

{\ttfamily
<{}< SERVER: auth: <<login|password>>

>{}> SERVER: success: <<auth>>
}

Если пара логин-пароль неверна, сервер отвечает пакетом \texttt{fail}. Если клиент уже авторизован, сервер отправляет в ответ пакет \texttt{wtf}.

\subsection{Регистрация}
\label{register}

Используется для создания новой учётной записи на сервере.

Процедура в общем случае выглядит так:

{\ttfamily
<{}< SERVER: register: <<login|password>>

>{}> SERVER: success: <<register>>
}

Если такой аккаунт уже существует или по какой-то другой причине выполнить регистрацию невозможно, сервер отвечает пакетом \texttt{fail}.

\section{Текстовые сообщения}

\subsection {Сообщение}
\label{message}

Используется для отправки обычных текстовых сообщений адресату.

{\ttfamily
<{}< friend: message: <<Привет!>>
}

Факт доставки никак не подтверждается.

\section{Работа со списком контактов}

\subsection{Запрос списка}

Запрашивает у сервера список контактов. Ответ приходит в формате INI. Заголовкам соответствуют группы контактов, ключам --- ники, значениям --- адреса.

{\ttfamily
<{}< SERVER: contacts-list: <<>>

>{}> SERVER: contacts-list: <<[General]

friend=friend@m1kc.tk

vasya=pupkin@poupkine.com

[Other]

number=one@m1kc.tk

>>
}

\subsection{Добавление контакта}

Для добавления контакта используется пакет типа \texttt{contacts-add}. В поле \texttt{Content} передаются id контакта и назначаемые ему ник и группа. Если заданной группы не существует, она будет создана.

Типичный случай:

{\ttfamily
<{}< SERVER: contacts-add: <<wellie@server.ru|One more friend|General>>

>{}> SERVER: success: <<contacts-add>>
}

\subsection{Переименование контакта}

Контакты переименовываются с помощью пакета типа \texttt{contacts-rename}. В поле \texttt{Content} передаются id контакта и его новый ник. Если такого контакта в списке нет, сервер отвечает пакетом \texttt{fail}.

{\ttfamily
<{}< SERVER: contacts-rename: <<friend@m1kc.tk|Enemy>>

>{}> SERVER: success: <<contacts-rename>>
}

\subsection{Удаление контакта}

Для удаления контакта используется пакет типа \texttt{contacts-remove}. В поле \texttt{Content} передаётся id контакта. Если такого контакта в списке нет, сервер отвечает пакетом \texttt{fail}.

{\ttfamily
<{}< SERVER: contacts-remove: <<one@m1kc.tk>>

>{}> SERVER: success: <<contacts-remove>>
}

\subsection{Получение списка групп}

Пакет типа \texttt{contacts-groups-list} используется для получения списка групп.

{\ttfamily
<{}< SERVER: contacts-groups-list: <<one@m1kc.tk>>

>{}> SERVER: contacts-groups-list: <<General|Other>>
}

Может, убрать его вообще?

\subsection{Переименование группы}

Пакет типа \texttt{contacts-groups-rename} используется для переименования группы. В поле \texttt{Content} передаются название группы и её новое имя. Если такой группы нет, сервер отвечает пакетом \texttt{fail}.

{\ttfamily
<{}< SERVER: contacts-groups-rename: <<General|Main>>

>{}> SERVER: success: <<contacts-groups-rename>>
}

Группа также удаляется автоматически, когда из неё удалён последний контакт. Напоминаем, что группа автоматически создаётся при добавлении контакта, если группы с заданным названием ещё нет.

\subsection{Удаление группы}

Пакет типа \texttt{contacts-groups-remove} используется для удаления группы и всех входящих в неё контактов. В поле \texttt{Content} передаётся название группы. Если такой группы нет, сервер отвечает пакетом \texttt{fail}.

{\ttfamily
<{}< SERVER: contacts-groups-remove: <<General>>

>{}> SERVER: success: <<contacts-groups-remove>>
}

\end{document}